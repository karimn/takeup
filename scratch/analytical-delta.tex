\documentclass{article}
\usepackage{amsmath}
\usepackage{amsfonts}
\begin{document}

\section*{Unbounded Type}

\subsection*{$\Delta^*(b,c)$ Derivation}
We want to find:

\begin{align*}
    \Delta^*(b,c) &= \frac{
        -1
    }{
        F_w(w^*(b,c)) [1 - F_w(w^*(b,c))]
    } \int^\infty_{-\infty} vF_u(w^*(b,c) - v) f_v(v) dv
\end{align*}

We suppress $w^*(b,c)$ notation and just write $w$. $V \sim N(0, 1), U \sim N(0, \sigma)$ 
and the two are independent.


Ignoring the first fraction:

\begin{align*}
    &\int vF_u(w^*(b,c) - v) f_v(v) dv \\
    &= 
    \int \underbrace{v \phi(v)}_{d(-\phi(v))} \Phi\left(\frac{w-v}{\sigma}\right) dv \\
    &= \int - \Phi\left(\frac{w-v}{\sigma}\right) d \phi(v) \\
    &= \left[
        -\phi(v) \Phi\left(\frac{w-v}{\sigma}\right) 
    \right]^\infty_{-\infty} - \int \frac{-1}{\sigma}\phi\left(\frac{w - v}{\sigma}\right) (-) \phi(v) dv \ \text{ (Integration by parts)} \\ 
    &= \left[
        -\phi(v) \Phi\left(\frac{w-v}{\sigma}\right) 
    \right]^\infty_{-\infty} -\frac{1}{\sigma} \int \phi\left(\frac{w - v}{\sigma}\right) \phi(v) dv 
\end{align*}


Now focus solely on the remaining integral:

\begin{align*}
\int \phi\left(\frac{w - v}{\sigma}\right) \phi(v) dv  &= 
\frac{1}{\sqrt{2\pi}} \frac{1}{\sqrt{2\pi}} 
\int \exp \left[
    -\frac{1}{2} 
    \left(
        \left(\frac{w - v}{\sigma}\right)^2 +
        v^2
    \right)
\right] dv \\
&= 
\frac{1}{\sqrt{2\pi}} \frac{1}{\sqrt{2\pi}} 
\int \exp \left[
    -\frac{1}{2} 
    \left(
        \left(\frac{w }{\sigma}\right)^2 +
        \left(\frac{v }{\sigma}\right)^2 
        -\left(\frac{2wv }{\sigma}\right)^2 +
        v^2
    \right)
\right] dv \\
&=
\frac{1}{\sqrt{2\pi}} \frac{1}{\sqrt{2\pi}} 
 \int \exp \left[
    -\frac{1}{2} 
    \left(
        \left(\frac{w }{\sigma}\right)^2 +
        v^2 \frac{\sigma^2 + 1}{\sigma^2}  
        -\left(\frac{2wv }{\sigma}\right)^2 
    \right)
\right] dv \\
&=
\frac{1}{\sqrt{2\pi}} \frac{1}{\sqrt{2\pi}} 
 \int \exp \left[
    -\frac{1}{2} 
    \left(
        \left(\frac{w }{\sigma}\right)^2 +
        \frac{\sigma^2 + 1}{\sigma^2} \left[
            \left(
                v - \frac{w}{\sigma^2 + 1}
            \right)^2 - \frac{w^2}{(\sigma^2 + 1)^2}
        \right]
    \right)
\right] dv \\ &\text{ (Completing  The Square)} \\
&=
\frac{1}{\sqrt{2\pi}} \frac{1}{\sqrt{2\pi}} 
 \int \exp \left[
    -\frac{1}{2} 
    \left(
        \frac{w^2 }{\sigma^2 + 1} +
        \frac{\sigma^2 + 1}{\sigma^2} 
            \left(
                v - \frac{w}{\sigma^2 + 1}
            \right)^2 
    \right)
\right] dv \  \\
&= 
\frac{1}{\sqrt{2\pi}} \frac{1}{\sqrt{2\pi}} 
 \exp \left(
    -\frac{1}{2} 
        \frac{w^2 }{\sigma^2 + 1}
         \right)
\int \exp\left[
    -\frac{1}{2}
        \frac{\sigma^2 + 1}{\sigma^2} 
            \left(
                v - \frac{w}{\sigma^2 + 1}
            \right)^2 
\right] dv \  \\
\end{align*}

The first exponential term is just a function of $w$ and $\sigma$ 
whilst the the second term is very nearly a normal pdf:

\begin{align*}
\frac{1}{\sqrt{2\pi}}
\int \exp\left[
    -\frac{1}{2}
        \frac{\sigma^2 + 1}{\sigma^2} 
            \left(
                v - \frac{w}{\sigma^2 + 1}
            \right)^2 
\right] dv &=  
                    \sqrt{
                        \frac{\sigma^2}{\sigma^2 + 1}
                    } \times
\frac{1}{\sqrt{2\pi}} 
\frac{1}{
                    \sqrt{
                        \frac{\sigma^2}{\sigma^2 + 1}
                    }
}
\int \exp\left[
    -\frac{1}{2}
            \left(
                \frac{v - \frac{w}{\sigma^2 + 1}}{
                    \sqrt{
                        \frac{\sigma^2}{\sigma^2 + 1}
                    }
                }
            \right)^2 
\right] dv  \\
&= 
                    \sqrt{
                        \frac{\sigma^2}{\sigma^2 + 1}
                    } 
\end{align*}

Putting this back together:
\begin{align*}
 \left[
        -\phi(v) \Phi\left(\frac{w-v}{\sigma}\right) 
    \right]^\infty_{-\infty} &-\frac{1}{\sigma} \int \phi\left(\frac{w - v}{\sigma}\right) \phi(v) dv  
     \\
    &=
 \left[
        -\phi(v) \Phi\left(\frac{w-v}{\sigma}\right) 
    \right]^\infty_{-\infty} -
    \frac{1}{\sigma} \frac{
        \exp \left[
            -\frac{1}{2}\frac{w^2}{\sigma^2 + 1}
        \right]
    }{\sqrt{2\pi}} \times 
                    \sqrt{
                        \frac{\sigma^2}{\sigma^2 + 1}
                    }  \\
&=
    \frac{-1}{\sigma} \frac{
        \exp \left[
            -\frac{1}{2}\frac{w^2}{\sigma^2 + 1}
        \right]
    }{\sqrt{2\pi}} \times 
                    \sqrt{
                        \frac{\sigma^2}{\sigma^2 + 1}
                    }  \\
\end{align*}


Plus the original fraction we ignored:

\begin{align*}
    \Delta^*(b,c) &= \frac{
        -1
    }{
        F_w(w^*(b,c)) [1 - F_w(w^*(b,c))]
    } \int^\infty_{-\infty} vF_u(w^*(b,c) - v) f_v(v) dv \\
     &= 
\frac{
        1
    }{
        \Phi\left(\frac{w}{\sqrt{\sigma^2 + 1}}\right)
         \left[1 - \Phi\left(\frac{
            w
        }{
            \sqrt{\sigma^2 + 1}
        }\right)\right]
    } \times 
    \frac{1}{\sigma} \frac{
        \exp \left[
            -\frac{1}{2}\frac{w^2}{\sigma^2 + 1}
        \right]
    }{\sqrt{2\pi}} \times 
                    \sqrt{
                        \frac{\sigma^2}{\sigma^2 + 1}
                    }  \\
\end{align*}

\subsection*{$\Delta'^*(b,c)$ Derivation}


\begin{align*}
   \Delta'[w] &= 
   \frac{
    \int^\infty_{-\infty} v f_u(w - v) f_v(v) dv 
    + f_w(w) \left[
        1 - 2 F_w(w)
    \right] \Delta[w]
   }{
    F_w(w)(1 - F_w(w))
   } \\
   &= 
   \frac{
    \int^\infty_{-\infty} v \frac{1}{\sigma}\phi\left(\frac{w - v}{\sigma}\right) \phi(v) dv 
    + \frac{1}{\sqrt{1 + \sigma^2}}\phi\left(\frac{w}{\sqrt{1 + \sigma^2}}\right) \left[
        1 - 2 \Phi(\frac{w}{\sqrt{1 + \sigma^2}})
    \right] \Delta[w]
   }{
    \Phi\left(\frac{w}{\sqrt{1 + \sigma^2}}\right)\left(1 - \Phi\left(\frac{1}{\sqrt{1 + \sigma^2}}\right)\right)
   } \\
\end{align*}


Focusing on the integral and recognising we've derived this above:

\begin{align*}
    \int^\infty_{-\infty} v \frac{1}{\sigma}\phi\left(\frac{w - v}{\sigma}\right) \phi(v) dv \\ &=
    \frac{1}{\sigma} \frac{\exp\left(-\frac{1}{2}\frac{w^2}{\sigma^2 + 1}\right)}{
        \sqrt{2\pi}
    }
    \Sigma \times 
    \frac{1}{\sqrt{2\pi}}\frac{1}{\Sigma} \int^\infty_{-\infty}
    v \exp\left(
        -\frac{1}{2}
        \left(\frac{v - \mu}{\Sigma}\right)^2
    \right) dv
\end{align*}

Where $\mu = \frac{w}{\sigma^2 + 1}, \Sigma = \sqrt{\frac{\sigma^2}{\sigma^2 + 1}}$.

Define:
\begin{align*}
    H = \frac{1}{\sigma} \frac{
        \exp\left( 
            -\frac{1}{2}\frac{w^2}{\sigma^2 + 1}
        \right)\Sigma
    }{\sqrt{2\pi}}
\end{align*}
Now perform change of variables:
\begin{align*}
    \frac{v - \mu}{\Sigma} &= y \\
    dy = dv \frac{1}{\Sigma}
\end{align*}

Giving:
\begin{align*}
    H \frac{1}{\sqrt{2\pi}} \frac{\Sigma}{\Sigma} \int^\infty_{-\infty} \left(
        \Sigma y + \mu
    \right)\exp\left(-\frac{1}{2}y^2\right) dy &= 
H\Sigma \left[
    -\phi(y)
\right]^\infty_{-\infty} + H\mu \\
&= H\mu
\end{align*}


Plugging this back into the formula:

\begin{align*}
   \Delta'[w] &= 
   \frac{
    H\mu
    + \frac{1}{\sqrt{1 + \sigma^2}}\phi\left(\frac{w}{\sqrt{1 + \sigma^2}}\right) \left[
        1 - 2 \Phi(\frac{w}{\sqrt{1 + \sigma^2}})
    \right] \Delta[w]
   }{
    \Phi\left(\frac{w}{\sqrt{1 + \sigma^2}}\right)\left(1 - \Phi\left(\frac{1}{\sqrt{1 + \sigma^2}}\right)\right)
   } \\
\end{align*}

\section*{Bounded Type}

Now we go back and bound types between $\underline{v}, \overline{v}$. There are 
three things we need to do here: keep any terms that dropped out due to integration 
limits being infinite before, replace the normal pdf with the truncated normal pdf
for $V$, and calculate the convolution of $W\sim V + U$ when $V$ is truncated normal.


\subsection*{$\Delta^*(w)$ Derivation}

\begin{align*}
    \Delta^*(w) &= \frac{-1}{F_w(w)(1 - F_w(w))} \frac{1}{\Phi(\overline{v}) - \Phi(\underline{v})} \times \left(
        \left[ -\phi(v)\Phi\left(\frac{w - v}{\sigma}\right)\right]^{\overline{v}}_{\underline{v}} 
        - \Gamma \left[
            \Phi\left(
                \frac{v - \frac{w}{\sigma^2 + 1}}{\frac{\sigma}{\sqrt{ \sigma^2 + 1}}}
            \right)
        \right]^{\overline{v}}_{\underline{v}}
    \right)
\end{align*}

Where:
\begin{align*}
    \Gamma &= \frac{1}{\sigma}\exp\left(
        -\frac{1}{2} \frac{w^2}{1 + \sigma^2}
    \right) \times \sqrt{\frac{\sigma^2}{\sigma^2 + 1}}
\end{align*}

and:
\begin{align*}
    F_w(w) = \int^{\overline{v}}_{\underline{v}} \Phi \left(
        \frac{w - t}{\sigma}
    \right) \frac{\phi(t)}{\Phi(\overline{v}) - \Phi(\underline{v})} dt  \\
    (\Phi(\overline{v}) - \Phi(\underline{v}))F_w(w) = \int^{\overline{v}}_{\underline{v}} \Phi \left(
        \frac{w - t}{\sigma}
    \right) \phi(t) dt 
\end{align*}

That is, the convolution of two independent r.v.s. After a lot of math it can be shown that the RHS:


\textbf{Note: We swap notation here because Ed is an idiot and came back to the 
bounded case after the unbounded case.}
Now: $U \sim N(0, \gamma), \mu = \gamma^{-1}z, \sigma = -\gamma^{-1}, \rho = \sqrt{1 + \sigma^2}, z_l = \frac{l - \mu}{\sigma}, z_h = \frac{h-\mu}{\sigma}, 
h = \frac{z - \overline{v}}{\gamma}, l = \frac{z - \underline{v}}{\gamma}$

\begin{align*}
    RHS &= \frac{1}{2} \left(
        \Phi(z_h) - \Phi(z_l)
    \right)  \\
    &- \frac{1}{2} \left[
        \frac{\mu}{z_h} < 0
    \right] \\
    &+ \frac{1}{2} \left[
        \frac{\mu}{z_l} < 0
    \right] \\
    &- T\left(z_h, \frac{h}{z_h}\right) \\
    &+ T\left(z_l, \frac{l}{z_l}\right) \\
    &- T\left(\frac{\mu}{\rho}, \frac{
        \mu\sigma + z_h \rho^2
    }{
        \mu
    }\right) \\
    &+ T\left(\frac{\mu}{\rho}, \frac{
        \mu\sigma + z_l \rho^2
    }{
        \mu
    }\right) \\
\end{align*}

We do the above because we need to perform change of variables so that we have:

\begin{align*}
    \int^h_l \Phi(x)\frac{1}{\sigma} \phi(\frac{x - \mu}{\sigma})dx
\end{align*}

So everything looks weird to transform the $\Phi(...)$ term into $\Phi(x)$ and we 
can use the known result above. $T$ stands for Owen's T.

\end{document}